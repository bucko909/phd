\documentclass{article} 
\usepackage{beamerarticle}

\mode<presentation>
{
  \usetheme{Warsaw}
  % or ...

  %\setbeamercovered{transparent}
  % or whatever (possibly just delete it)
}

\usepackage{amsmath}
\usepackage{amssymb}
\usepackage{amsthm}
\usepackage{color}
\usepackage{epsfig}

\usepackage[english]{babel}
% or whatever

\usepackage[latin1]{inputenc}
% or whatever

\usepackage{times}
\usepackage[T1]{fontenc}
% Or whatever. Note that the encoding and the font should match. If T1
% does not look nice, try deleting the line with the fontenc.


%\title[Conjugacy in Hyperbolic Groups] % (optional, use only with long paper titles)
%{Title As It Is In the Proceedings}

\title{Conjugacy in Hyperbolic Groups}

%\author[Author, Another] % (optional, use only with lots of authors)
%{F.~Author\inst{1} \and S.~Another\inst{2}}
% - Give the names in the same order as the appear in the paper.
% - Use the \inst{?} command only if the authors have different
%   affiliation.
\author{David Buckley}

%\institute[Universities of Somewhere and Elsewhere] % (optional, but mostly needed)
%{
%  \inst{1}%
%  Department of Computer Science\\
%  University of Somewhere
%  \and
%  \inst{2}%
%  Department of Theoretical Philosophy\\
%  University of Elsewhere}
% - Use the \inst command only if there are several affiliations.
% - Keep it simple, no one is interested in your street address.

% - Either use conference name or its abbreviation.
% - Not really informative to the audience, more for people (including
%   yourself) who are reading the slides online

% This is only inserted into the PDF information catalog. Can be left
% out. 



% If you have a file called "university-logo-filename.xxx", where xxx
% is a graphic format that can be processed by latex or pdflatex,
% resp., then you can add a logo as follows:

%TODO
%\pgfdeclareimage[height=1cm]{university-logo}{the_warwick_uni_blue}
%\logo{\pgfuseimage{university-logo}}



% Delete this, if you do not want the table of contents to pop up at
% the beginning of each subsection:
%\AtBeginSubsection[]
%{
%  \begin{frame}<beamer>{Outline}
%    \tableofcontents[currentsection,currentsubsection]
%  \end{frame}
%}


% If you wish to uncover everything in a step-wise fashion, uncomment
% the following command: 

%\beamerdefaultoverlayspecification{<+->}


\begin{document}

\begin{frame}
  \titlepage
\end{frame}

%\begin{frame}{Outline}
%  \tableofcontents
  % You might wish to add the option [pausesections]
%\end{frame}


% Structuring a talk is a difficult task and the following structure
% may not be suitable. Here are some rules that apply for this
% solution: 

% - Exactly two or three sections (other than the summary).
% - At *most* three subsections per section.
% - Talk about 30s to 2min per frame. So there should be between about
%   15 and 30 frames, all told.

% - A conference audience is likely to know very little of what you
%   are going to talk about. So *simplify*!
% - In a 20min talk, getting the main ideas across is hard
%   enough. Leave out details, even if it means being less precise than
%   you think necessary.
% - If you omit details that are vital to the proof/implementation,
%   just say so once. Everybody will be happy with that.

\section{Ingredients}

\subsection{A Hyperbolic Group}

\begin{frame}{Hyperbolic Space}{Subtitles are optional.}
  % - A title should summarize the slide in an understandable fashion
  %   for anyone how does not follow everything on the slide itself.

  \begin{itemize}
  	\item Hyperbolic spaces curve "inwards."
	\item This makes geodesic triangles thin.

  \only<1-2>{\uncover<2>{\begin{figure}
  \input thin_triangle.pstex_t
  \end{figure}}}

  \only<3-4>{\begin{figure}
  \input thin_triangle2.pstex_t
  \end{figure}}

  \uncover<4>{\item Hyperbolic if all geodesic triangles are $\delta$-thin.
  }
  \end{itemize}

\end{frame}

\begin{frame}{Hyperbolic Groups}

  \begin{itemize}
  \item The Cayley graph of a finitely generated group is a geodesic metric space.
  \item A group is hyperbolic if its Cayley graph is hyperbolic.
  \item Examples:
  \begin{itemize}
    \pause
    \item Finite groups.
    \pause
    \item Free groups of finite rank.
    \pause
    \item Groups acting properly and cocompactly on hyperbolic spaces.
  \end{itemize}
  \pause
  \item Non-example: $\mathbb{Z}\times\mathbb{Z}$
  \pause
  \item Fix $G = <\!\!X | R\!\!>$ a finitely generated group with $\delta$-hyperbolic Cayley graph $\Gamma$.
  \item Let $A = X \cup X^{-1}$.
  \end{itemize}

\end{frame}


\subsection{A Quasiconvex Subgroup}

\begin{frame}{Quasiconvex Subgroups}
  \begin{itemize}
    \item Let $w$ be a word in $A^*$.
	\item Suppose $w$ labels a geodesic $\alpha$ in $\Gamma$ starting at $1$.
	\item Let $H < G$ any subgroup.
	\pause
	\item $d(w, H)$ is the max distance from a point on $\alpha$ to closest point in $H$.
  	\begin{figure}
	\input quasiconvex.pstex_t
	\end{figure}
	\pause
	\item $H$ $\epsilon$-quasiconvex if $d(w, H) \le \epsilon$ for any such $w \in H$.
	\pause
	\item For instance, $<\!\!a^2\!\!>$ is a $2$-quasiconvex subgroup of $F(a, b)$.
  \end{itemize}
\end{frame}

\subsection{Some Elements}

\begin{frame}{The Problem}
  \begin{itemize}
    \item We're given a word $u \in A^*$.
	\item In time $O(|u|)$, we must:
	\pause
	\begin{itemize}
	  \item Check if there exists $w \in A^*$ such that $wuw^{-1} \in H$.
	  \pause
	  \item Return such a $w$, if one exists.
	\end{itemize}
	\pause
	\item We assume $G$ and $H$ are fixed.
	\item Exponential running time with respect to $\delta + \epsilon$.
  \end{itemize}
\end{frame}

\section{Equipment Required}

\subsection{Shortest Words}

\begin{frame}
  \begin{itemize}
    \item Given any word $w \in A^*$, find some $w' \in A^*$ such that:
	\begin{itemize}
	  \item $w' =_G w$, and
	  \item $|w'|$ is minimal
	\end{itemize}
	\pause
	\item Shapiro's algorithm in ``The linearity of the conjugacy problem in word hyperbolic groups'' (Epstein+Holt, 2006).
	\item Runs in time linear in $O(|w|)$.
	\pause
	\item Assume that \textbf{all} words are shortest.
	\item If not, just run Shapiro's algorithm.
  \end{itemize}
\end{frame}

\subsection{Minimal Conjugates}

\begin{frame}{Minimal Conjugates}
  \begin{itemize}
    \item Given any word $u \in A^*$, find some $u', w \in A^*$ such that:
	\begin{itemize}
	  \item $u' =_G w^{-1}uw$, and
	  \item $|u'|$ is minimal
	\end{itemize}
	\pause
  	\item Strictly speaking, do not have this.
	\pause
	\item Do have an algorithm computing this for $u^M$ where $M$ is some constant.
	\item Sufficient; compute with $u^M$, then use centraliser of $u^M$ to check everything for $u$.
	\pause
	\item We therefore assume $u$ in the problem is its own minimal conjugate.
  \end{itemize}
\end{frame}

\subsection{Solution to Conjugacy Problem}

\begin{frame}{The Conjugacy Problem}
  \begin{itemize}
    \item Given any words $u, v \in A^*$:
	\begin{itemize}
	  \item Check if there is a $g \in A^*$ with $u =_G g^{-1}vg$.
	  \item Return some such $g$ if one exists. 
    \end{itemize}
	\pause
	\item Solved by Epstein and Holt (2006).
	\item Runs in time $O(|u| + |v|)$.
  \end{itemize}
\end{frame}

\section{Recipe}

\subsection{Idea}

\begin{frame}{Idea}
  \begin{itemize}
    \item Suppose $g \in G$ has $gug^{-1} \in H$ as required.
	\pause
	\item Pick $h \in H$ and $w \in A^*$ such that $w =_G hg$ and $|w|$ is minimal.
	\item Then $wuw^{-1} = v \in H$.
	\item Aim is to either:
	\pause
	\begin{itemize}
	  \item find an $O(1)$ bound on the length of $w$, or
	  \pause
	  \item find an $O(1)$ bound on the length of $v$
	\end{itemize}
	\pause
	\item Bounded number of checks taking time $O(|u|)$ means $O(|u|)$ time.
	\item Upper bounds will all depend on constants $\delta$ and $\epsilon$.
  \end{itemize}
\end{frame}

\subsection{Bounding $|w|$}

\begin{frame}{Bounding the Number of $w$}
  \begin{itemize}
    \item Suppose a point on $u$ corresponds to a point on $v$.
	\pause
	\item $m$ and $n$ distances to the meeting points on $w$ from ends of $u$.
	\begin{figure}
	\input conj1_mn.pstex_t
	\end{figure}
	\pause
	\item If $\min\{m,n\} > \delta$ there is a suffix $w'$ of $w$ such that $|wuw^{-1}| < |u|$ (contradiction!).
  \end{itemize}
\end{frame}

\begin{frame}{Bounding the Number of $w$}
  \begin{itemize}
	\item $p$ and $q$ distances to meeting points on $w$ from ends of $v$.
	\only<1-2,4->{
	\begin{figure}
	\input conj1_pq.pstex_t
	\end{figure}
	}
	\only<3>{
	\begin{figure}
	\input conj1_pq2.pstex_t
	\end{figure}
	}
	\pause
	\item If $\max\{p,q\} > 2\delta + \epsilon$ there is a word $w' \in A^*$ such that $|w'| < |w|$ and $hw' =_G w$ for some $h \in H$ (contradiction!).
	\pause
	\pause
	\item $|w| = m + p = n + q$, so $|w| \le 3\delta + \epsilon$.
	\pause
	\item $|w| \in O(1)$ and there $O(1)$ possibilities for $w$.
	\pause
	\item Check if $wuw^{-1} \in H$ for all $|w| \le 3\delta + \epsilon$.
  \end{itemize}
\end{frame}

\subsection{Bounding $|v|$}

\begin{frame}
  \begin{itemize}
    \item Suppose no point on $u$ corresponds to a point on $v$.
	\pause
	\begin{figure}
	\input conj2.pstex_t
	\end{figure}
	\pause
	\item By the same argument as before, $\max\{p,q\} \le 2\delta + \epsilon$.
	\pause
	\item $|v| = p + q \le 4\delta + 2\epsilon$.
	\pause
	\item So $|v| \in O(1)$ and there are $O(1)$ possibilities for $v$.
	\pause
	\item For each $|v| \le 4\delta + 2\epsilon$, check if there is $w$ such that $wuw^{-1} = v$.
  \end{itemize}
\end{frame}

\section{Finished Product}

\begin{frame}{Finished Product}
  \begin{itemize}
    \item Can with some more work extend this to checking if, given a list $a_1, \ldots, a_n$, there is one $w$ such that $wa_iw^{-1} \in H$ for all $i$.
	\pause
    \item Useful because if there is, $<\!\!a_1, \ldots, a_n\!\!>$ conjugates into $H$.
  \end{itemize}
  \pause
\end{frame}
  
\end{document}



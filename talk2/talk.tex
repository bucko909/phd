% $Header: /cvsroot/latex-beamer/latex-beamer/solutions/conference-talks/conference-ornate-20min.en.tex,v 1.7 2007/01/28 20:48:23 tantau Exp $

\documentclass{beamer}

\mode<presentation>
{
  \usetheme{Warsaw}
  % or ...

  %\setbeamercovered{transparent}
  % or whatever (possibly just delete it)
}


\usepackage[english]{babel}
% or whatever

\usepackage[latin1]{inputenc}
% or whatever

\usepackage{times}
\usepackage[T1]{fontenc}
% Or whatever. Note that the encoding and the font should match. If T1
% does not look nice, try deleting the line with the fontenc.


%\title[] % (optional, use only with long paper titles)
%{Title As It Is In the Proceedings}

\title{Subgroups of Hyperbolic Groups}

\subtitle{A Default Beamer Presentation}

%\author[Author, Another] % (optional, use only with lots of authors)
%{F.~Author\inst{1} \and S.~Another\inst{2}}
% - Give the names in the same order as the appear in the paper.
% - Use the \inst{?} command only if the authors have different
%   affiliation.
\author{David Buckley}

%\institute[Universities of Somewhere and Elsewhere] % (optional, but mostly needed)
%{
%  \inst{1}%
%  Department of Computer Science\\
%  University of Somewhere
%  \and
%  \inst{2}%
%  Department of Theoretical Philosophy\\
%  University of Elsewhere}
% - Use the \inst command only if there are several affiliations.
% - Keep it simple, no one is interested in your street address.
\institute[University of Warwick]{
  Mathematics Department\\
  University of Warwick
}

\date[PGTC 2009] % (optional, should be abbreviation of conference name)
{Postgraduate Group Theory Conference, 2009}
% - Either use conference name or its abbreviation.
% - Not really informative to the audience, more for people (including
%   yourself) who are reading the slides online

\subject{Computational Group Theory}
% This is only inserted into the PDF information catalog. Can be left
% out. 



% If you have a file called "university-logo-filename.xxx", where xxx
% is a graphic format that can be processed by latex or pdflatex,
% resp., then you can add a logo as follows:

%TODO
%\pgfdeclareimage[height=1cm]{university-logo}{the_warwick_uni_blue}
%\logo{\pgfuseimage{university-logo}}
%\pgfdeclareimage[height=5cm]{portalcake}{portalcake}



% Delete this, if you do not want the table of contents to pop up at
% the beginning of each subsection:
%\AtBeginSubsection[]
%{
%  \begin{frame}<beamer>{Outline}
%    \tableofcontents[currentsection,currentsubsection]
%  \end{frame}
%}


% If you wish to uncover everything in a step-wise fashion, uncomment
% the following command: 

%\beamerdefaultoverlayspecification{<+->}


\begin{document}

\begin{frame}
  \titlepage
\end{frame}

%\begin{frame}{Outline}
%  \tableofcontents
  % You might wish to add the option [pausesections]
%\end{frame}


% Structuring a talk is a difficult task and the following structure
% may not be suitable. Here are some rules that apply for this
% solution: 

% - Exactly two or three sections (other than the summary).
% - At *most* three subsections per section.
% - Talk about 30s to 2min per frame. So there should be between about
%   15 and 30 frames, all told.

% - A conference audience is likely to know very little of what you
%   are going to talk about. So *simplify*!
% - In a 20min talk, getting the main ideas across is hard
%   enough. Leave out details, even if it means being less precise than
%   you think necessary.
% - If you omit details that are vital to the proof/implementation,
%   just say so once. Everybody will be happy with that.

\section{Introduction}

\subsection{Cayley Graphs}

\begin{frame}{Cayley Graphs}
	\begin{itemize}
		\item Group $G$ generated by set $X$.
		\item Cayley graph of $G$ is $Cay(G, X)$ \pause :
		\item Vertex-set $G$, edge set $\{(g, gx) : g \in G, x \in X\}$.
		\pause
		\item Subgroup $H \le G$.
		\item Coset Cayley graph of $G$ relative to $H$\pause:
		\item Vertex-set $\{Hg : g \in G\}$, edge set $\{(Hg, Hgx) : g \in G, x \in X\}$.
	\end{itemize}
\end{frame}

\subsection{Hyperbolic Groups}

\begin{frame}{Hyperbolic Space}
  % - A title should summarize the slide in an understandable fashion
  %   for anyone how does not follow everything on the slide itself.

  \begin{itemize}
  	\item Hyperbolic spaces curve "inwards."
	\item This makes geodesic triangles thin.

  \only<1-2>{\uncover<2>{\begin{figure}
  \input thin_triangle.pstex_t
  \end{figure}}}

  \only<3-4>{\begin{figure}
  \input thin_triangle2.pstex_t
  \end{figure}}

  \uncover<4>{\item Hyperbolic if all geodesic triangles are $\delta$-thin.
  }
  \end{itemize}

\end{frame}

\begin{frame}{Divergence}
	\begin{itemize}
		\item Metric space $X$.
		\item A map $e : [0, \infty] \to [0, \infty]$ is a divergence function if:
		\begin{itemize}
			\item When $r_1, r_2 \subset X$ two rays which start at the same point
			\item and $d(r_1(a), r_2(a)) > e(0)$ for some $a>0$
			\item then $d_{a+b}(r_1(a+b), r_2(a+b)) > e(b)$ for all $b>0$ ($d_l$ means distance outside of $B_l(1)$).
		\end{itemize}
		\pause
		\item Hyperbolic spaces have exponential divergence functions.
		\pause
		\item Will assume divergence is strict:
		\item Then $e(b) < d_{a+b}(r_1(a+b), r_2(a+b)) < ke(b)$ for all $b > 0$.
	\end{itemize}
\end{frame}

\begin{frame}{Hyperbolic Groups}
  \begin{itemize}
  \item The Cayley graph of a finitely generated group is a geodesic metric space.
  \item A group is hyperbolic if its Cayley graph is hyperbolic.
  \item Examples:
  \begin{itemize}
    \pause
    \item Finite groups.
    \pause
    \item Free groups of finite rank.
    \pause
    \item Groups acting properly and cocompactly on hyperbolic spaces.
  \end{itemize}
  \pause
  \item Non-example: $\mathbb{Z}\times\mathbb{Z}$
  \pause
  \item Fix $G = <\!\!X | R\!\!>$ a finitely generated group with $\delta$-hyperbolic Cayley graph $\Gamma$.
  \item Let $A = X \cup X^{-1}$.
  \end{itemize}

\end{frame}

\subsection{Quasiconvexity}

\begin{frame}{Quasiconvex Subgroups}
  \begin{itemize}
    \item Let $w$ be a word in $A^*$.
	\item Let $\alpha$ be the path starting at $1$ and labelled by $w$.
	\item Let $H < G$ any subgroup.
	\item $d(w, H)$ is the max distance from a point on $\alpha$ to closest point in $H$.
  	\begin{figure}
	\input quasiconvex.pstex_t
	\end{figure}
	\pause
	\item $H$ $\epsilon$-quasiconvex if $d(w, H) \le \epsilon$ for any $w \in H$ which labels a geodesic.
	\pause
	\item For instance, $<\!\!a^2\!\!>$ is a $1$-quasiconvex subgroup of $F(a, b)$.
  \end{itemize}
\end{frame}

\subsection{Ends}

\begin{frame}{Cut points}
  \begin{itemize}
    \item Connected topological space $X$
	\item Some point $x \in X$
	\item $x$ is a cut point if $X \backslash \{ x \}$ is disconnected
	\item Example: Every point in $\mathbb{R}$ is a cut point
	\pause
	\item Subset $Y \subset X$
	\item $Y$ is a cut set if $X \backslash Y$ is disconnected
	\item Example: Every closed 1-ball in $\mathbb{R} \times [0,1]$ is a cut set.
	\pause
	\item $Y$ cuts set $A$ from set $B$ if $A \backslash Y$ and $B \backslash Y$ lie in different components of $X \backslash Y$.
  \end{itemize}
\end{frame}

\begin{frame}{Ends}
	\begin{itemize}
		\item Metric space $X$
		\item Geodesic rays $r_1, r_2$ (path isometric to $[0, \infty]$)
		\item Say $r_1 \equiv r_2$ if there is no bounded set $Y$ which cuts $r_1$ from $r_2$.
		\item Space of ends is space of equivalence classes of rays $r$.
		\pause
		\item Compare to boundary - equivalence classes of geodesic rays under $r \equiv s$ if there exists $M$ such that $d(x, s) \le M$ for all $x \in r$ and vice versa.
	\end{itemize}
\end{frame}

\begin{frame}{Number of Ends}
	\begin{itemize}
		\item Interested in \textsl{number} of ends n(X).
		\item For a Cayley graph there are 0, 1, 2 or $\infty$ ends. %TODO ref
		\pause
		\item Example: Finite groups have 0-ended Cayley graphs (there are no geodesic rays).
		\pause
		\item Example: Hyperbolic triangle groups have 1-ended Cayley graphs (since they represent tilings of hyperbolic 2-space).
		\pause
		\item Example: $\mathbb{Z}$ has a 2-ended Cayley graph (there's precisely 2 directions).
		\pause
		\item Example: Non-cyclic free groups have $\infty$-ended Cayley graphs (any string of positive powers like $abb$ can be repeated indefinitely to give a unique end).
	\end{itemize}
\end{frame}

\section{Ends of the Coset Cayley Graph}

\begin{frame}{Problem}
	\begin{itemize}
		\item $G$ is a hyperbolic group with Cayley graph $\Gamma$.
		\item $G$ has 1 end.
		\item $H$ is a quasiconvex subgroup.
		\item $\Gamma$ has a strict divergence function $e$.
		\item What can we say about ends of coset Cayley graph?
	\end{itemize}
\end{frame}

\begin{frame}{Example}
	\begin{itemize}
		\item Let $G$ be the hyperbolic triangle group $\langle x, y|x^2, y^3, (xy)^7 \rangle$.
		\item Let $H = \langle xyxy^{-1} \rangle$ (infinite, infinite index).
		\pause
		\begin{figure}
		\input cosetexample.pstex_t
		\end{figure}
	\end{itemize}
\end{frame}

\begin{frame}{A Neat Trick}
	\begin{itemize}
		\item Suppose $w$, $a$ and $b$ are words.
		\item Also, that $w$ is long enough (about $\epsilon+2\delta$).
		\pause
		\item Suppose $wa$ does not have much cancellation
		\item (Roughly, $|wa| > |w| + |a| - k$).
		\pause
		\item Then $Hwa$ does not have much cancellation.
		\item (Roughly, $d(H, Hwa) > d(H, Hw) + |a| - k - \epsilon - 2\delta$).
	\end{itemize}
\end{frame}

\begin{frame}{Divergence Made Useful}
	\begin{itemize}
		\item Divergence is a bit of a pain to use, but strict divergence is (obviously) easier!
		\item Suppose $d(r_1(a), r_2(a)) > e(0)$ with $a$ minimal.
		\item If $\alpha$ is a path connecting $r_1(a+b)$ to $r_2(a+b)$, then there exists a word $w$ of length nearly $a$ and near-suffixes $r_1' =_G w^{-1}r_1$ and $r_2' =_G w^{-1}r_2$ such that:
		\pause
		\item $wr_1'$ and $wr_2'$ don't have much cancellation
		\item (Roughly $e^{-1}(k)$ where $k$ is the $k$ you got along with $e$).
		\pause
		\item for all paths $x$ to points in $\alpha$, there is $x' =_G w^{-1}x$ such that $wx'$ don't have much cancellation.
	\end{itemize}
\end{frame}

\begin{frame}{Wave Hands = Proof}
	\begin{itemize}
		\item So there's only a finite number of ends of $G$ relative to $H$.
		\pause
		\item Proof? You want proof?
		\pause
		\only<3>{
			\item Take a huge ball $B_K(H)$.
  			\begin{figure}
			\input proof1.pstex_t
			\end{figure}
		}
		\only<4>{
			\item Let $Hx, Hy$ be points in not in the ball, and in infinite components.
  			\begin{figure}
			\input proof2.pstex_t
			\end{figure}
		}
		\only<5>{
			\item Since they're in infinite components, may as well assume $d(Hx, H) \ge 2K$ and $d(Hy, H) \ge 2K$.
  			\begin{figure}
			\input proof3.pstex_t
			\end{figure}
		}
		\only<6>{
			\item There's a path $\alpha$ connecting $x$ to $y$ in the group Cayley graph
  			\begin{figure}
			\input proof4.pstex_t
			\end{figure}
		}
		\only<7>{
			\item There's a magical path $w$ which doesn't cancel too much with $x$, $y$, or paths from $1$ to points on $\alpha$.
  			\begin{figure}
			\input proof5.pstex_t
			\end{figure}
		}
		\only<8>{
			\item If $w$ is big enough, all points $Hx$ on $H\alpha$ have $d(H, Hx)$ a bit lower than $K$. Thus $H\alpha$ connects $Hx$ and $Hy$ and lies outside the ball.
			\item Basically, this means the component a point $Hx$ lies in is determined completely by its behaviour inside a small ball around $H$.
			\item There's only a finite number of short prefix words, so only a finite number of ends.
		}
	\end{itemize}
\end{frame}

\end{document}
